% Options for packages loaded elsewhere
\PassOptionsToPackage{unicode}{hyperref}
\PassOptionsToPackage{hyphens}{url}
%
\documentclass[
]{article}
\usepackage{amsmath,amssymb}
\usepackage{lmodern}
\usepackage{iftex}
\ifPDFTeX
  \usepackage[T1]{fontenc}
  \usepackage[utf8]{inputenc}
  \usepackage{textcomp} % provide euro and other symbols
\else % if luatex or xetex
  \usepackage{unicode-math}
  \defaultfontfeatures{Scale=MatchLowercase}
  \defaultfontfeatures[\rmfamily]{Ligatures=TeX,Scale=1}
  \setmainfont[]{SourceSansPro}
\fi
% Use upquote if available, for straight quotes in verbatim environments
\IfFileExists{upquote.sty}{\usepackage{upquote}}{}
\IfFileExists{microtype.sty}{% use microtype if available
  \usepackage[]{microtype}
  \UseMicrotypeSet[protrusion]{basicmath} % disable protrusion for tt fonts
}{}
\makeatletter
\@ifundefined{KOMAClassName}{% if non-KOMA class
  \IfFileExists{parskip.sty}{%
    \usepackage{parskip}
  }{% else
    \setlength{\parindent}{0pt}
    \setlength{\parskip}{6pt plus 2pt minus 1pt}}
}{% if KOMA class
  \KOMAoptions{parskip=half}}
\makeatother
\usepackage{xcolor}
\usepackage[margin=1in]{geometry}
\usepackage{graphicx}
\makeatletter
\def\maxwidth{\ifdim\Gin@nat@width>\linewidth\linewidth\else\Gin@nat@width\fi}
\def\maxheight{\ifdim\Gin@nat@height>\textheight\textheight\else\Gin@nat@height\fi}
\makeatother
% Scale images if necessary, so that they will not overflow the page
% margins by default, and it is still possible to overwrite the defaults
% using explicit options in \includegraphics[width, height, ...]{}
\setkeys{Gin}{width=\maxwidth,height=\maxheight,keepaspectratio}
% Set default figure placement to htbp
\makeatletter
\def\fps@figure{htbp}
\makeatother
\setlength{\emergencystretch}{3em} % prevent overfull lines
\providecommand{\tightlist}{%
  \setlength{\itemsep}{0pt}\setlength{\parskip}{0pt}}
\setcounter{secnumdepth}{-\maxdimen} % remove section numbering
\ifLuaTeX
  \usepackage{selnolig}  % disable illegal ligatures
\fi
\IfFileExists{bookmark.sty}{\usepackage{bookmark}}{\usepackage{hyperref}}
\IfFileExists{xurl.sty}{\usepackage{xurl}}{} % add URL line breaks if available
\urlstyle{same} % disable monospaced font for URLs
\hypersetup{
  pdftitle={Практическая работа №7. Быстрое умножение теплицевой и циркулянтной матрицы на вектор},
  hidelinks,
  pdfcreator={LaTeX via pandoc}}

\title{Практическая работа №7. Быстрое умножение теплицевой и
циркулянтной матрицы на вектор}
\author{}
\date{\vspace{-2.5em}2022-11-25}

\begin{document}
\maketitle

\hypertarget{ux446ux438ux440ux43aux443ux43bux44fux442ux43dux430ux44f-ux43cux430ux442ux440ux438ux446ux430}{%
\section{\texorpdfstring{\textbf{Циркулятная
матрица}}{Циркулятная матрица}}\label{ux446ux438ux440ux43aux443ux43bux44fux442ux43dux430ux44f-ux43cux430ux442ux440ux438ux446ux430}}

Циркулянтная матрица - вид матрицы, получающеся в результате смещения
влево первой строки матрицы на один элемент с переносом крайнего
элемента на первую позицию справа в вектор-строке матрицы A.
Итерационный перенос всей строки в рамках матрицы для каждой строки дает
нам возможность строить циркулянтную матрицу:

\[
A^{CIRC} = \begin{pmatrix} a_0 & a_1 & a_2 & \dots & a_{n-1} \\
a_{n-1} & a_0 & a_1 & \dots & a_{n-2} \\ 
a_{n-2} & a_{n-1} & a_0 & \dots & a_{n-3} \\
\vdots & \vdots & \vdots & \ddots & \vdots \\
a_1 & a_2 & a_3 & \dots & a_{0}
\end{pmatrix}
\]

Данная матрица обладает свойством периодичности, в результате которой
имеется несколько новых возможностей:

\begin{enumerate}
\def\labelenumi{\arabic{enumi}.}
\item
  Не хранить матрицу полностью в памяти комптьютера, а получать её
  генератором, посредством повторного переноса первой строки.
\item
  Умножать такую матрицу на вектор не за \(O(n^2)\), а за
  \(O(n\cdot log_2(n)\) операций с помощью быстрого преобразования
  Фурье.
\end{enumerate}

Во втором случае имеет место следующее равенство:

\[
A^{CIRC} \ @ \ u = f,\quad F^{-1}\left[\ F[A^{CIRC}_{0}] \cdot F[u]\ \right] \approx f,
\] где \(F[\cdot]\ -\) быстрое преобразование Фурье входного вектора,
\(F^{-1}[\cdot]\ -\) быстрое обратное преобразование Фурье входного
вектора, \(A \ @ \ b\ -\) матричное умножение матрицы на вектор по
правилам линейной алгебры, \(a \cdot b\ -\) поэлементное умножения
векторов, \(A^{CIRC}_{0}\ -\) первая строка циркулянтной матрицы
\(A^{CIRC}\).

Полученные выше выражения для умножения матрицы на вектор являеются
идентичными с точностью до ошибок округления компьютерной арифметики и
второй способ является эффективным методом умножения циркулянтной
матрицы на вектор за короткое время.

\hypertarget{ux442ux435ux43fux43bux438ux446ux435ux432ux430-ux43cux430ux442ux440ux438ux446ux430}{%
\section{\texorpdfstring{\textbf{Теплицева
матрица}}{Теплицева матрица}}\label{ux442ux435ux43fux43bux438ux446ux435ux432ux430-ux43cux430ux442ux440ux438ux446ux430}}

\[
A^{TOEPL} = 
\begin{pmatrix} 
a_0 & a_1 & a_2 & \dots & a_{n-1} \\
a_{-1} & a_0 & a_1 & \dots & a_{n-2} \\ 
a_{-2} & a_{-1} & a_0 & \dots & a_{n-3} \\
\vdots & \vdots & \vdots & \ddots & \vdots \\
a_{-n+1} & a_{-n+2} & a_{-n+3} & \dots & a_{0}
\end{pmatrix}
\]

\hypertarget{ux43fux43eux441ux442ux430ux43dux43eux432ux43aux430-ux437ux430ux434ux430ux447ux438}{%
\section{\texorpdfstring{\textbf{Постановка
задачи}}{Постановка задачи}}\label{ux43fux43eux441ux442ux430ux43dux43eux432ux43aux430-ux437ux430ux434ux430ux447ux438}}

\end{document}
